\documentclass[runningheads]{llncs}


\usepackage{graphicx}
\usepackage{tikz}
\usepackage{pgfplots}
\usepackage{xcolor}
\usepackage{subcaption}
\usepackage{todonotes}
\pgfplotsset{compat=1.18} 
\usepackage[acronym]{glossaries}

\authorrunning{Silva, Sousa, Santos, Costa, Araújo, Formosinho and Cabrita}

\newacronym{es}{ES}{Expert Systems}
\newacronym{ml}{ML}{Machine Learning}
\newacronym{cip}{CIP}{continuous improvement process}

\bibliographystyle{splncs04}

\begin{document}

\title{State of the Art Review for Autonomous Fire Detection and Suppression}

\author{João Silva\inst{1} \and Ricardo Sousa\inst{1} \and
João Santos\inst{1} \and Nuno Costa\inst{1} \and\\ José Araújo\inst{1} \and Diogo Formosinho\inst{1} \and Francisco Cabrita\inst{1}}

\institute{Instituto Superior de Engenharia do Porto\\
\email{\{1150425, 1160900, 1161023, 1171584,\\ 1180943, 1210056, 1210058\}@isep.ipp.pt}}

\maketitle

\begin{abstract}

This article presents an overview of the state of the art regarding the application of machine learning powered robotic apparatus and multi agent systems for fire detection and suppression in an industrial context.

\todo{Finish abstract}

\keywords{Fire Detection  \and Fire Suppression \and Marketing \and Sentiment Analysis \and Product Development.}

\end{abstract}

\section{Introduction}
\label{sec:int}

All industrial operations are founded on a production model, whereas each independent operation produces a given product \cite{industrymwd}. Specifically, we see this model applied in natural resource extraction, where a given industrial operation extracts a raw material from the environment, creating value from said extraction.

\todo{add wood industry size statistics}

One such example is the lumber industry, which focuses on the process of obtaining wood from trees. This process is usually destructive and requires special attention to sustain a stable raw material output. In order to do this, these operations usually undertake constructive actions to plant trees, in order to sustain a tree production cycle of 10 years or more that guarantees a steady availability of trees suitable for extraction \cite{lumber}.

These actions however require however not only an initial effort but a continuous attention to the state of the environment in order to guarantee and further improve the yield of said actions. Attention must be paid to, for example, fire risks within the managed timber land. \todo{add statistic regarding forest loss to fires}

In order to detect these fires or other situations that may pose a risk to the forest, aerial surveillance is usually employed, in order to quickly scan large swaths of land. \todo{include data regarding usage of planes vs uav?}

With the advent of the Industry 4.0, this data gathering can be increased significantly in both volume and scope, further improving its usefulness and availability \cite{Hood_Brady_2016}.

With this increase in data availability, new methodologies and heuristics are required in order properly capture and process said data in a timely manner. This article provides an overview of the state of the art for UAV usage in the forestry industry, autonomous multi-agent fire detection and suppression systems, while taking into account the expected problems (and possible solutions) that the stated problem may encounter.

\subsection{Autonomous UAVs}


\subsection{Fire Detection}



\subsection{Fire Suppression}


\subsection{Autonomous UAVs in Foresting}


\section{Areas of Interest}

All of the aforementioned topics have areas of interest in terms of application techniques, field presence and research direction. We present the most common fields where these topics are applied and the research options for the topic, taking into account the most recent developments in usage and research.

\subsection{Autonomous UAVs}


\subsection{Fire Detection}



\subsection{Fire Suppression}


\subsection{Autonomous UAVs in Foresting}


\section{Academic Production}

Most of the topics mentioned in this article have seen steady or increased interest, as measured by the number of research papers published. This allows us to envision a future with more and better options to solve problems that integrate these topics as available solution paths.

\subsection{Autonomous UAVs}

\subsection{Fire Detection}

\subsection{Fire Suppression}

\section{State of the Art}

\subsection{Autonomous UAVs}

\subsection{Fire Detection}

\subsection{Fire Suppression}

\section{Applications}

\subsection{Autonomous UAVs}

\subsection{Fire Detection}

\subsection{Fire Suppression}

\section{Conclusion}

\bibliography{refs}

\end{document}