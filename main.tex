\documentclass[runningheads]{llncs}


\usepackage{graphicx}
\usepackage{tikz}
\usepackage{pgfplots}
\usepackage{xcolor}
\usepackage{subcaption}
\usepackage{todonotes}
\pgfplotsset{compat=1.18} 
\usepackage[acronym]{glossaries}

\authorrunning{Silva, Sousa, Santos, Costa, Araújo, Formosinho and Cabrita}

\newacronym{uav}{UAV}{Unmanned Aerial Vehicle}

\bibliographystyle{splncs04}

\begin{document}

\title{State of the Art Review for Autonomous Fire Detection and Suppression}

\author{João Silva\inst{1} \and Ricardo Sousa\inst{1} \and
João Santos\inst{1} \and Nuno Costa\inst{1} \and\\ José Araújo\inst{1} \and Diogo Formosinho\inst{1} \and Francisco Cabrita\inst{1}}

\institute{Instituto Superior de Engenharia do Porto\\
\email{\{1150425, 1160900, 1161023, 1171584,\\ 1180943, 1210056, 1210058\}@isep.ipp.pt}}

\maketitle

\begin{abstract}

This article presents an overview of the state of the art regarding the application of machine learning powered robotic apparatus and multi agent systems for fire detection and suppression in an industrial context.

\todo{Finish abstract}

\keywords{Fire Detection  \and Fire Suppression.}

\end{abstract}

\section{Introduction}
\label{sec:int}

All industrial operations are founded on a production model, whereas each independent operation produces a given product \cite{industrymwd}. Specifically, we see this model applied in natural resource extraction, where a given industrial operation extracts a raw material from the environment, creating value from said extraction.

\todo{add wood industry size statistics}

One such example is the lumber industry, which focuses on the process of obtaining wood from trees. This process is usually destructive and requires special attention to sustain a stable raw material output. In order to do this, these operations usually undertake constructive actions to plant trees, in order to sustain a tree production cycle of 10 years or more that guarantees a steady availability of trees suitable for extraction \cite{lumber}.

These actions however require however not only an initial effort but a continuous attention to the state of the environment in order to guarantee and further improve the yield of said actions. Attention must be paid to, for example, fire risks within the managed timber land.

In order to detect these fires or other situations that may pose a risk to the forest, aerial surveillance is usually employed, in order to quickly scan large swaths of land. \todo{include data regarding usage of planes vs uav?}

With the advent of the Industry 4.0, this data gathering can be increased significantly in both volume and scope, further improving its usefulness and availability \cite{Hood_Brady_2016}.

With this increase in data availability, new methodologies and heuristics are required in order properly capture and process said data in a timely manner. This article provides an overview of the state of the art for \acrshort{uav} usage in the forestry industry, autonomous multi-agent fire detection and suppression systems, while taking into account the expected problems (and possible solutions) that the stated problem may encounter.

\subsection{Autonomous \acrshort{uav}}

\acrshort{uav} were arguably first developed in Austria during the blockade of the Republic of Venice in 1849. During this blockade, Austrian forces launched several unnamed balloons carrying timed explosives \todo{cite future of drone use by bart custers, page 355}. From that point on, various developments have improved the usability and viability of \acrshort{uav} usage, including the advent of ground controlled aerial vehicles and increased technological capabilities in these vehicles. In this article we will provide a brief overview of the state of the art for \acrshort{uav}, with a special focus on existing developments into autonomous vehicles.

\subsection{Fire Detection}

Fire is a menace to the well being of managed forests, and it's late detection can make fire suppression arduous or even impossible, which leads to damage to the environment and atmosphere, as well as further long term consequences. \todo{add statistic regarding forest loss to fires}

As such, early detection is extremely important in order to maintain a healthy forest ecosystem. In this article, we will provide an overview on the state of the art techniques used to quickly and accurately detect fires in natural environments.

\subsection{Fire Suppression}

As mentioned previously, fire suppression is extremely important to sustain a forest ecosystem as part of an industrial lumber operation. Furthermore, the development of the state of a fire can lead to increase cost to suppress said fire, including subsequent costs related to loss of materials \todo{get citation}. In this article we will provide an overview on techniques and tools used for quickly and efficiently suppress fires.

\subsection{Autonomous \acrshort{uav} in Foresting}

Finally, we also research and provide examples of autonomous \acrshort{uav} usage within the foresting industry, both as fire suppression and detection tools and on other tasks within the industry.

\section{Areas of Interest}

All of the aforementioned topics have areas of interest in terms of application techniques, field presence and research direction. We present the most common fields where these topics are applied and the research options for the topic, taking into account the most recent developments in usage and research.

\subsection{Autonomous \acrshort{uav}}


\subsection{Fire Detection}



\subsection{Fire Suppression}


\subsection{Autonomous \acrshort{uav} in Foresting}


\section{Academic Production}

Most of the topics mentioned in this article have seen steady or increased interest, as measured by the number of research papers published. This allows us to envision a future with more and better options to solve problems that integrate these topics as available solution paths.

\subsection{Autonomous \acrshort{uav}}

\subsection{Fire Detection}

\subsection{Fire Suppression}

\section{State of the Art}

\subsection{Autonomous \acrshort{uav}}

\subsection{Fire Detection}

\subsection{Fire Suppression}

\section{Applications}

\subsection{Autonomous \acrshort{uav}}

\subsection{Fire Detection}

\subsection{Fire Suppression}

\section{Conclusion}

\bibliography{refs}

\end{document}